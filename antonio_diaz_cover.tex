\documentclass[12pt,a4paper]{article}
\usepackage{currvita}
\usepackage[english]{babel}
\usepackage{microtype}
\usepackage[left=3cm, right=3cm, top=3cm]{geometry} % Margins
\usepackage{graphicx}
\usepackage[T1]{fontenc}
\usepackage{cfr-lm}
\usepackage{eso-pic}
\usepackage{array}
\usepackage{xcolor}
\usepackage[overload]{textcase}
\usepackage{regexpatch}
\usepackage[unicode, hidelinks]{hyperref}
\newcommand*{\ac}[1]{\mbox{#1}}
\pagenumbering{gobble} % No page numbers
%% Labels right alignment %%
\tracingxpatches
\makeatletter
\xpatchcmd{\cvlist}{##1\hfill}{\hfill##1}{}{\ddt}
\makeatother
%% Labels right alignment %%
\tolerance=600

\begin{document}
    \begin{cv}
    {\fontsize{20pt}{16pt}\selectfont
    {\bfseries\sc\huge{Antonio Díaz}}\\\\
    {\mdseries\it\normalsize{Software developer}}\\
    {\sc\mdseries\normalsize{\href{https://www.typescriptlang.org}{TypeScript} + \href{https://www.rust-lang.org/}{Rust}}}}
        \vspace{4em}
        Software developer currently focused on \href{https://www.rust-lang.org/}{Rust}, with 5+ years of experience {\mbox{primarily}} on frontend using \href{https://www.typescriptlang.org}{TypeScript} —\href{https://reactjs.org/}{React}/\href{https://angular.io/}{Angular}— and on backend with \href{https://nodejs.org/en/}{NodeJS} and {\href{https://www.python.org/}{Python}} —\href{https://www.djangoproject.com/}{Django}—.\\

        Since 2015 I have been working on full stack roles related to web software \mbox{development} with multinational teams across wide time zones. Some sites I have been involved in are the ecommerce based in Los Angeles \href{https://www.thrivemarket.com}{\textit{Thrive Market}} or the site for the Russian design studio \href{https://linii.group}{\textit{Linii}}.\\

        I also have background in art and design industries, focusing my attention on {\mbox{digital}} technologies. I worked as designer and editor at the contemporary art magazine \href{https://www.revistasculturales.com/revistas/119/art-notes/numeros/}{\textit{\mbox{Art Notes}}} and with several institutions as project coordinator, as \href{https://cgac.xunta.gal/}{\sc{cgac}} or Medialab-Prado —currently \href{https://www.medialab-matadero.es/}{\it{Matadero-Medialab}}—. On 2012 I co-founded \href{https://www.antoniodiaz.me/diazpons-catalog-2016.pdf}{\it{\mbox{Díaz \& Pons}}}, publishing house focused on digital and printed editions, where we edited high quality non-fiction books on art and social sciences. I founded and developed Critik in 2015, an online platform for literary enthusiasts created for Eidos Editorial, migrating {\sc{rdf}} data from a Virtuoso server using {\sc{sparql}} into our own {\sc{m}}y{\sc{sql}} database.\\

        As a former publisher I focus on readable and well-structured code, but also on methodologies that allow fluent and flexible procedures with complex teams while maintaining on-time delivery. In this regard I am currently interested in strongly typed languages such as \href{https://www.rust-lang.org/}{Rust} and design approaches that involve both technical and non-technical members of the teams.\\

        I consider roles related to web software development with \href{https://www.typescriptlang.org}{TypeScript}, \href{https://nodejs.org/en/}{NodeJS} or \href{https://www.rust-lang.org/}{Rust}. You can find more information at my site {\href{https://antoniodiaz.me}{\textit{www.antoniodiaz.me}}}, in my GitLab at {\href{https://www.git.antoniodiaz.me/antoniodcorrea/}{\textit{www.git.antoniodiaz.me}}}, or just writing me to {\href{mailto:hello@antoniodiaz.me?subject=Hello!&body=...}{\textit{hello@antoniodiaz.me}}}.\\\\\\\\

        Antonio Díaz

        \cvplace{}
        \date{November 22\textsuperscript{nd} 2022}
      \end{cv}
\end{document}
\endinput
%%
%% End of file `cvtest.tex'.
